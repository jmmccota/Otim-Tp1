\documentclass [11pt]{articleSBPO}

\usepackage[brazil]{babel}
\usepackage[utf8]{inputenc}
\usepackage[T1]{fontenc}
\usepackage{graphics}
\usepackage[a4paper,top=3.3cm,left=2.9cm,right=2.9cm,bottom=2.5cm,noheadfoot]{geometry}
\usepackage{algorithm}
\usepackage{algorithmic}
\usepackage{times}
\usepackage{amsmath}
\usepackage{amssymb}
\usepackage{setspace}
\usepackage{graphicx}
\usepackage{subfig}
\usepackage{indentfirst}
\usepackage{icomma}
\usepackage{url}
\usepackage{longtable}
\usepackage{lscape}
\usepackage{array}
\usepackage[alf]{abntex2cite}
\usepackage{epstopdf}
\usepackage{rotating}
\newcommand{\up}[1]{\raisebox{1.3ex}[0pt]{#1}}
% \usepackage{latex8}
\floatname{algorithm}{Algoritmo}
\def\figurename{Figura}
\def\tablename{Tabela}


%     configurações do pacote algorithm
% \renewcommand{\algorithmcfname}{alg}
\floatname{algorithm}{Algoritmo}
\renewcommand{\algorithmicrequire}{\textbf{Entrada:}}
\renewcommand{\algorithmicensure}{\textbf{Saída:}}
\renewcommand{\algorithmicend}{\textbf{fim}}
\renewcommand{\algorithmicif}{\textbf{se}}
\renewcommand{\algorithmicthen}{\textbf{então}}
\renewcommand{\algorithmicelse}{\textbf{senão}}
\renewcommand{\algorithmicelsif}{\algorithmicelse\ \algorithmicif}
\renewcommand{\algorithmicendif}{\algorithmicend\ \algorithmicif}
\renewcommand{\algorithmicfor}{\textbf{para}}
% \renewcommand{\algorithmicto}{\textbf{até}}
\renewcommand{\algorithmicforall}{\textbf{para todo}}
\renewcommand{\algorithmicdo}{\textbf{faça}}
\renewcommand{\algorithmicendfor}{\algorithmicend\ \algorithmicfor}
\renewcommand{\algorithmicwhile}{\textbf{enquanto}}
\renewcommand{\algorithmicendwhile}{\algorithmicend\ \algorithmicwhile}
\renewcommand{\algorithmicloop}{\textbf{laço}}
\renewcommand{\algorithmicendloop}{\algorithmicend\ \algorithmicloop}
\renewcommand{\algorithmicrepeat}{\textbf{repita}}
\renewcommand{\algorithmicuntil}{\textbf{até que}}
\renewcommand{\algorithmicprint}{\textbf{imprima}}
\renewcommand{\algorithmicreturn}{\textbf{retorne}}
\renewcommand{\algorithmictrue}{\textbf{verdadeiro}}
\renewcommand{\algorithmicfalse}{\textbf{falso}}
\renewcommand{\algorithmicnot}{\textbf{não}}

%\usepackage{attrib}

%\usepackage{natbib} % pacote que traz o formato das citações por autor (não por números, que é o default do LaTeX)

\newtheorem{Lemma}{Lemma}
\newtheorem{Theorem}{Theorem}
\newtheorem{Condition}{Condition}

\def\nohyphen{\pretolerance=1000 \tolerance=1000 \hyphenpenalty=1000 \exhyphenpenalty=1000}

\setlength{\parindent}{1.50cm}












%%%%%%%%%%%%%%%%%%%%%%%%%%%%%%%%%%%%%%%%%%%%%%%%%%%%%%%%%%%%%%%%%%%%%%%%%%%%%%%%%%%%%%%%%%%%%%%%%%%%%%
%%											 ATUALIZAR												%%
\newcommand{\sigla}[1] {SIMPL}
\newcommand{\nome}[1] {Sistema Interativo para Métodos de Programação Linear}
%%%%%%%%%%%%%%%%%%%%%%%%%%%%%%%%%%%%%%%%%%%%%%%%%%%%%%%%%%%%%%%%%%%%%%%%%%%%%%%%%%%%%%%%%%%%%%%%%%%%%%

\begin{document}
\pagestyle{empty}%tira a numeração das páginas

\thispagestyle{empty}%tira a numeração da página em questão

\begin{center}
\LARGE{\textbf{\normalsize DESENVOLVIMENTO DE UMA FERRAMENTA PARA ENSINO INTERATIVO DE OTIMIZAÇÃO}}
\end{center}

\vspace{5mm}

%%%%%%%%%%%%%%%%%%%%%%%%%%%%%%%%%%%%%%%%%%%%%%%%%%%%%%%%%%%%%%%%%%%%%%%%%%%%%%%%%%%%%%%%%%%%%%%%%%%%%%
%%											 ATUALIZAR												%%
\begin{center}
\textbf{André F. R. Malta$^{\alpha}$, Daniel G. Oliveira$^{\alpha}$, Elias L. da S. Júnior$^{\alpha}$, João M. M. da C. Cota$^{\alpha}$, André R. da Cruz$^{\beta}$} \\
Centro Federal de Educação Tecnológica de Minas Gerais (CEFET-MG)  Campus Timóteo, \\
Rua 19 de Novembro, 121 – Centro, Timóteo - MG, Brasil \\
\{andrmalta, eliasluizjr\}@gmail.com \\
\{, joao\_marcos\_cota\}@hotmail.com.com \\
\par
$\alpha$ Graduando em Engenharia da Computação \\
$\beta$  \\ 
\end{center}
%%%%%%%%%%%%%%%%%%%%%%%%%%%%%%%%%%%%%%%%%%%%%%%%%%%%%%%%%%%%%%%%%%%%%%%%%%%%%%%%%%%%%%%%%%%%%%%%%%%%%%


\begin{center}
{\bf RESUMO}
\end{center}
\nohyphen{ 


\noindent \textbf{PALAVRAS CHAVE. } 

\vspace{11pt}

\noindent \textbf{Áreas Principais: }
}

\begin{center}
{\bf ABSTRACT}
\end{center}
\nohyphen{


\noindent \foreignlanguage{english}{\textbf{KEYWORDS. }}

\vspace{11pt}

\noindent \foreignlanguage{english}{\textbf{Main areas: }}
}

\newpage

\section{Introdução}\label{sec:introducao}


\section{Fundamentação Teórica}\label{sec:fundamentacao}


\subsection{Simplex}\label{subsec:simplex}

\subsection{Branch and Bound}\label{subsec:bnb}

\subsection{Algoritmos de Transporte}\label{subsec:transporte}

\subsubsection{Método do Canto Noroeste}\label{subsubsec:noroeste}

\subsubsection{Método do Menor Custo}\label{subsubsec:menorcusto}

\subsubsection{Método da Aproximação de Vogel}\label{subsubsec:mav}


\section{Desenvolvimento}\label{sec:desenvolvimento}


\section{Resultados}\label{sec:resultados}


\section{Conclusões}\label{sec:conclusao}



\bibliography{referencias}
\bibliographystyle{abnt-alf}

\end{document}